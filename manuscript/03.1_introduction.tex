
\section{Introduction}
% \quickwordcount{03.1_introduction}

A Monte Carlo simulation is a computational technique employed to simulate the probability of various outcomes in complex systems or processes where predictions are challenging due to the involvement of random variables, or the absence of analytical models. Monte Carlo simulation has been widely applied in the field of magnetic resonance imaging (MRI) across various applications (e.g. \cite{harkins2021simple, baxter2013computational, storey2024signatures, soher2001representation, garpebring2013uncertainty, prange2009quantifying}). Particularly noteworthy is its application in modeling functional MRI contrast (\cite{baez2017impact, khajehim2017investigating, martindale2008theory}) and investigating diffusion phenomena within microstructure (\cite{rafael2020robust, yeh2013diffusion, lee2021realistic}), where Monte Carlo simulation has garnered significant interest since the complexity of the systems prevents the derivation of analytical solutions to generate ground truth data.


Monte Carlo approaches allow inclusion of various biological characteristics and configurations, such as microvascular networks (\cite{baez2017impact}), intravascular magnetic susceptibility (\cite{boxerman1995mr}), transverse relaxation (\cite{weisskoff1994microscopic}), mixture of axonal radii (\cite{lee2020impact}) or cell membrane permeability (\cite{yeh2013diffusion}) into the simulation. However, achieving stable and reliable simulation results requires a sufficiently large number of samples both for spins as random walkers and time steps (\cite{fieremans2018physical}). Also,  an accurate representation of the diffusion media's geometry and the physical properties of the underlying biological tissue or structure are crucial. Therefore, it is important to carefully consider available computation power and optimize the Monte Carlo simulation program accordingly. To mitigate computational burden, simplifications are often implemented, including simple measures such as using a smaller number of random walkers and time steps, a limited sample size, simple microgeometry, or more sophisticated approaches like confining 3D diffusion to 2D scenarios.


When subjected to magnetic field inhomogeneity, such as that found in BOLD fMRI, spins (here the protons of water molecules) experience off-resonance and accumulate phase, a process known as dephasing. The local field inhomogeneity, caused by susceptibility differences between blood and surrounding tissue, varies with the oxygenation level of blood. Moreover, spins may diffuse in all directions within the imaging voxel, experiencing different local field perturbations over time. Consequently, the dephasing rate can vary spatially and temporally. As the formation of MR signal includes the vector sum of all individual water proton magnetization within the imaging voxel, each with Brownian microscopic motion and a certain accumulated phase, signal attenuation occurs. This effect can also be observed in non-BOLD fMRI studies. For instance, recent work has explored how susceptibility-induced internal gradients within the extra-axonal and myelin compartments influence the pulsed gradient spin-echo (PGSE) signal (\cite{winther2024susceptibility}).s

Studies have evaluated various MR sequences in terms of their sensitivity to intravascular magnetic susceptibility changes in microvascular networks. These investigations have demonstrated strong agreement between Monte Carlo simulations and experimental findings. Specifically, the relationship between BOLD sensitivity and/or vessel size and MR signal contrast has been explored in gradient echo and spin echo sequences (\cite{baez2017impact, boxerman1995mr}), as well as in steady-state sequences such as bSSFP (\cite{baez2017impact, bieri2007effect}), FISP, and PSIF (\cite{scheffler2019bold, khajehim2017investigating}), and CPMG and GRASE echo trains (\cite{scheffler2021bold}).

Robust and reliable Monte Carlo simulations require using a very high number of spins and time steps; otherwise, convergence may not be achieved (\cite{hall2009convergence}). Especially in so-called steady-state sequences involving preparation phases, the number of time steps that need to be simulated significantly increases compared to for instance a computationally rather simple GRE sequence. This challenge is further stressed when simulating substrates with long \(T_1\) values, as the preparation steps become longer. For instance, as a rule of thumb, the time for a bSSFP sequence to reach the steady state is approximately 5 times the \(T_1\) value (\cite{bieri2013fundamentals}). Achieving a steady state for gray matter with a \(T_1\) value in the range of 2 seconds at ultrahigh-field MRI requires over 1000 RF pulses (with \(TR\) $<$ 10\,ms). This emphasises why a high-performance simulation tool is required to be able to lift restrictions on the type of sequence that can be investigated. Monte Carlo simulation of such scenarios is computationally expensive, and depending on the number of spins and available computing power, programs written in commonly used prototyping scientific languages like MATLAB and Python can take from several hours to a day to complete execution. If the contribution of a group of parameters such as blood volume, oxygenation level, or \(FA\) and \(TR\) needs to be evaluated separately, the simulation needs to be repeated from the beginning, which can extend over several days or even a week in total. Therefore, there is a need for a versatile framework that can execute such simulations optimally within a reasonable time frame.


Several open-source, community-developed tools exist for performing Monte Carlo simulations in the context of diffusion MRI. Notable examples include MC/DC (\cite{rafael2020robust}), MCMRSimulator (\cite{cottaar_2023_10067194}), Camino (\cite{cook2006camino}), Disimpy (\cite{kerkela2020disimpy}), RMS (\cite{lee2021realistic}), MATI (\cite{Xu2024mati}), and simDRIFT (\cite{blum2023simulated}). It is worth mentioning the matrix formalism method (\cite{callaghan1997simple}) as a computationally efficient alternative to the Monte Carlo approach for diffusion simulations in well-known geometrical models. MISST (\cite{ianucs2016microstructure}) is a widely used simulator of this type. While these tools offer advanced features for simulating various aspects of diffusion, they may not be fully equipped for simulations of BOLD fMRI. Some of limitations are: 1) simulating only DWI sequences with diffusion sensitization gradients, 2) simplifying by excluding off-resonance effects and/or tissue relaxations, 3) inability to incorporate microvascular networks into simulations, and 4) computational speed constraints that limit their usability for a wide range of sequences.

Below, we present our implementation of the Monte Carlo simulator, SpinWalk. SpinWalk is a high-performance simulator developed in C++ and harnessing CUDA technology to leverage GPU acceleration. In the absence of a GPU device, SpinWalk can seamlessly switch to a CPU-only mode. The core simulation routines are designed to exploit massively parallel processing capabilities. While this work primarily focuses on applying SpinWalk to BOLD contrast in functional MRI, and it is the main motivation to develop this software, SpinWalk is not confined to this application. It serves as a versatile framework for conducting realistic, rapid, and precise Monte Carlo simulations of spin diffusion. As such, it can be utilized for Monte Carlo simulation across various aspects of diffusion MRI too. It not only enables simulation of DW-SE sequences with PGSE (\cite{stejskal1965spin}) or oscillating gradient spin-echo (OGSE) (\cite{does2003oscillating}) settings, but also allows DW-SSFP simulations (\cite{mcnab2010steady}). Table \ref{tab:simulators} provides a comparison between the aforementioned simulators and SpinWalk. To the best of our knowledge, there are currently no publicly available simulators specifically designed for performing Monte Carlo simulations of BOLD fMRI. The publicly available projects we identified are primarily focused on simulating diffusion-weighted MR signals using diffusion encoding gradients, typical of PGSE experiments. Except for MCMRSimulator (\cite{cottaar_2023_10067194}), effects such as off-resonance, relaxation, and RF excitation are excluded for simplicity, limiting flexibility to arbitrary gradient waveforms. SpinWalk allows for an arbitrary number of substrates within the phantom, each with distinct relaxation constants, diffusivity, and permeability, and supports an arbitrary number of RFs and gradients with various shapes.


%TC:ignore 
\begin{landscape}
% Please add the following required packages to your document preamble:
% \usepackage{multirow}
\begin{table*}[t]%
\caption{list of some of open-source simulators for diffusion MRI, including their programming languages, licenses, supported computing units, and key features. The listed features for each simulator is derived from information available in their repositories, manuals, or the original publications introducing each simulator.} \label{tab:simulators}
\renewcommand{\arraystretch}{1.5} % Adjust cell height
\begin{threeparttable}
\begin{tabular}{cM{3cm}ccM{8cm}c}
\hline
\multicolumn{1}{c}{\textbf{Project Name}} & \textbf{Programming Language} & {\textbf{License}} & {\textbf{CPU/GPU}} & {\textbf{Features}} & {\textbf{Reference\tnote{1}}} \\  \hline
MC/DC & C++ & LGPL-2.1 & CPU & arbitrary gradients, save trajectory, triangle mesh phantom &  2c2ed25 \\
Disimpy & Python & MIT  & GPU & arbitrary gradients, save trajectory, triangle mesh phantom &  948c439 \\
MCMRSimulator & Julia & Apache 2.0 & CPU & arbitrary gradients \& RF, relaxation, off-resonance, Membrane permeability, thorough documentation &  1eab9ef6 \\
MISST & MATLAB & Artistic 2.0 &  CPU & arbitrary gradients & 0.95 \\
simDRIFT & Python & BSD-3-Clause & \ GPU & Gradients, NIfTI I/O, save trajectory & 15756bb \\
SpinWalk & C++ & MIT & CPU \& GPU & arbitrary gradients \& RF, multi-echo, relaxation, off-resonance, save trajectory, HDF5 I/O, voxel mesh phantom, permeability &  1.15 \\ 
\hline
\end{tabular}
\begin{tablenotes}%%[341pt]
\item[1] commit ID, released data , or released version
\end{tablenotes}
\end{threeparttable}
\end{table*}
\end{landscape}
%TC:endignore 


SpinWalk is designed to use a predefined numerical phantom. Phantoms with simple structures such as parallel cylinders or spheres can be generated with SpinWalk. Phantoms with more complex structure may be generated with tools like MEDUSA (\cite{ginsburger2019medusa}), CACTUS (\cite{villarreal2023cactus}), ConFiG (\cite{callaghan2020config}), etc. The process of determining the MR signal can be divided into two main stages: first, the creation of a phantom that represents the microstructural geometry and the corresponding off-resonance maps; and second, the execution of the random walk simulations based on the provided phantom. SpinWalk focuses on performing the random walk and simulating MR sequences but not on creating a phantom. Apart from the aforementioned toy examples, it is assumed that the microstructures and off-resonance maps are pre-generated and provided as three-dimensional matrices, which serve as input for SpinWalk. In this sense, any sophisticated model may be integrated into SpinWalk. If a complex microvascular model incorporating network dynamics, such as those proposed in \cite{baez2023mechanistic} and \cite{hartung2021mathematical}, is available, SpinWalk can be used to produce $\Delta$CBF, $\Delta$CBV, and $\Delta$SO2 contrast - of course dependent on what was implemented in the model - and distinguish different signals within a laminar profile. SpinWalk is open source, and we encourage contributions from the community via the project's GitHub repository at \url{https://github.com/aghaeifar/spinwalk}. 

