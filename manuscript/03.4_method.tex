\section{Methods}
% \quickwordcount{03.4_method}

The performance and capabilities of SpinWalk are demonstrated by reproducing results from previous works (\cite{baez2017impact, scheffler2019bold, scheffler2021bold, boxerman1995mr, bieri2007effect, kim2012analysis, uludaug2009integrative}). All simulations were conducted on a desktop PC equipped with an AMD Ryzen Threadripper PRO 5995WX CPU (2.70 GHz base frequency, 64 cores, 2 threads per core) and an NVIDIA RTX 6000 GPU (compute capability 8.9, 48 GB GDDR6 DRAM), unless stated otherwise.


\subsection*{Cylinder Model}
The utilization of cylinder-based models to represent vessels is widespread due to their adaptable nature, facilitating exploration of parameters such as volume fraction, vessel radius, and their alignment relative to \(B_0\). Furthermore, this approach offers an analytical solution for determining the resulting magnetic field inhomogeneities, making it especially suitable for simulating BOLD responses. The field inhomogeneities induced by an infinite cylinder can be expressed as (\cite{ogawa1993functional}):
\begin{equation}
\Delta B_z(r) = \begin{cases}2\pi B_0(1-Y)\Delta\chi(\frac{R}{r})^{2}\cos(2\phi)\sin^{2}\theta & r\ge R\\2\pi B_0(1-Y)\Delta\chi\left(\cos^{2}\theta-\frac{1}{3}\right) & r<R\end{cases}
\label{eq:cylinder_b0}
\end{equation}

In the equation, \(\Delta B_z\) represents the change in the magnetic field relative to \(B_0\), while \(\Delta\chi\) denotes the susceptibility difference between fully deoxygenated blood and tissue. The variable \(Y\) controls the blood oxygenation level, ranging from 0 to 1, with 1 indicating fully oxygenated. Additionally, \(R\) stands for the radius of the vessel, \(r\) corresponds to the Euclidean distance from the spin location to the axis of the cylinder, \(\phi\) represents the angle between \(r\) and the projection of \(B_0\) onto the plane orthogonal to the cylinder axis, and \(\theta\) signifies the angle between the cylinder axis and \(B_0\). Given the FoV and grid size, SpinWalk can create a phantom in which a specified fraction of the volume is occupied by randomly positioned, infinitely long cylinders of a certain radius and compute the corresponding field map according to equation \eqref{eq:cylinder_b0}. Note that modeling intravascular field inhomogeneities with equation \eqref{eq:cylinder_b0} may not be fully accurate due to the dynamic nature of blood cells, which act as sources of strong local dipolar fields. These cells are in constant motion, changing shape and exchanging water molecules (\cite{gagnon2016validation}). However, for the purposes of demonstration and to maintain consistency with the studies being reproduced, equation \eqref{eq:cylinder_b0} is still employed to estimate intravascular field inhomogeneities in this work.
uuu

\subsection*{Virtual FoV Scaling}
In the case of spatially isotropic oriented cylinders, e.g. all randomly or all parallel to each other, the evaluation of different vessel sizes can be efficiently conducted using a single numerical phantom. This approach eliminates the need to generate multiple phantoms and avoids the necessity to determine different fieldoffset maps by scaling the FoV and voxel resolution appropriately. For instance, a phantom with a FoV of 500\,µm and a grid size of 1000 (equals a voxel mesh resolution of 0.5\,µm), containing a cylinder with a radius of 10\,µm, can be alternatively used as a phantom with a FoV of 1000\,µm with a voxel mesh resolution of 1.0\,µm, containing a cylinder with a radius of 20\,µm. The scaling is achived by modifying step width in the diffusion process such that the cylinders appear to be smaller or bigger compared to the original step size. The fieldmap calculated using equation \eqref{eq:cylinder_b0} for the original phantom is applicable to the scaled phantom. This methodology avoids creation of numerous environments containing different vessel sizes and maintains a consistent volume fraction across simulations with different vessel sizes. SpinWalk is capable of processing an array of FoV scaling factors given in the configuration file and executing identical Monte Carlo simulations for a single numerical phantom across various FoV scalings. 

To evaluate the accuracy and stability of the FoV scaling approach, numerical phantoms with a FoV of 1000 \(\times\) 1000 \(\times\) 1000\,µm\textsuperscript{3} and a grid size of 1000 in each dimension were generated, each containing cylinders with a volume fraction of 4\%. The cylinder radii were chosen from the set [1, 2, 5, 8, 12, 20, 35, 50, 75]\,µm. For each cylinder radius, 10 phantoms with randomly positioned cylinders were created, resulting in a total of 90 numerical phantoms. Signal change using the GRE sequence, as described in subsection \textit{Simulations}, were simulated in each phantom while it was virtually scaled to mimic a phantom with a cylinder radius of 8\,µm. The mean and confidence intervals of 95\% (= mean $\pm$ 1.96 $\times$ std / $\sqrt{\text{10}} $) for each radius were calculated as criteria for assessment.

 The relationship between the original phantom's FoV and cylinder size with virtual FoV scaling was analyzed by repeating experiments on phantoms with FoVs of 2000\,µm and 4000\,µm, each containing cylinders with a radius of 75\,µm. Additionally, the impact of voxel mesh resolution was assessed using a phantom with a 500\,µm FoV filled with cylinders of 1 µm radius. This was done by creating phantoms with grid sizes of [250, 500, 1000, 2000], corresponding to voxel mesh resolutions of [2, 1, 0.5, 0.25]\,µm.

\subsection*{Simulations}
\label{subsec:Simulations}

Phantoms with a FoV of 600 \(\times\) 600 \(\times\) 600\,µm\textsuperscript{3} and a grid size of 1200 were generated for sequence simulation. They were filled with cylinders of 8 µm radius, and virtual FoV scaling was applied when simulating different vessel sizes. BOLD sensitivity was assessed as a function of vessel size, vessel orientation, and blood volume fraction (BVF). For vessel size and orientation examinations, the volume fraction of cylinders was set to 4\%, for BVF assessments, it ranged from 1\% to 6\%. All cylinders were aligned perpendicular to the B\textsubscript{0} direction, except in the vessel orientation assessment, where the angle between the cylinders and B\textsubscript{0} varied linearly from 0\textdegree to 90\textdegree.


Unless otherwise stated, all simulations with cylinder models were performed using the following parameters: B\textsubscript{0} = 9.4\,T, T\textsubscript{1} [IV, EV] = [2500, 2200]\,ms; T\textsubscript{2} [IV\textsubscript{Y=78\%}, IV\textsubscript{Y=85\%}, EV] = [13, 20, 41]\,ms (\cite{khajehim2017investigating}); oxygenation level \(Y\) [rest, active] = [78, 85]\%; \(\Delta\chi\) = 0.11 ppm in cgs units; number of spins = 5 \(\times\) 10\textsuperscript{5} ; time step = 50\,µs; diffusion constant [IV, EV] = 1.0\,µm\textsuperscript{2}/ms, and impermeable surfaces. Additionally, a simulation was conducted in a phantom with a BVF of 4\%, featuring permeable surfaces that allow water to cross vessel boundaries into tissue with a 10\% probability. Spins were constrained to remain within the FoV; if they attempted to exit, their direction of movement was reversed to keep them inside. 


The evaluations were conducted using the following sequences and changes in signal percentages (1 -S\textsubscript{rest}/S\textsubscript{active}) or signal differences (S\textsubscript{active} - S\textsubscript{rest}) were computed for extravascular contributions: 

\begin{itemize}
    \item Gradient Recalled Echo (GRE): TE = 20\,ms.
    \item Spin Echo (SE): TE = 30\,ms.
    \item balanced Steady State Free Precession (bSSFP): TE/TR = 5/10\,ms, FA = 20\textdegree, phase cycling = 180\textdegree, number of preparation pulse = 1100.
    \item GRAdient and Spin Echo (GRASE) (\cite{feinberg1991grase, constable1994functional}): nine spin-echos with 40\,ms echo spacing and one gradient-echo in between each, $90^\circ_x$ excitation RF followed by train of perfect $180^\circ_{\pm y}$ refocusing RF. 
    \item STimulated Echo (STE): a $90^\circ_x$ excitation RF pulse was followed by two $90^\circ_y$ refocusing RF pulses. The initial refocusing RF pulse was applied at 20\,ms, and the mixing time (T\textsubscript{m}, the duration between the refocusing pulses) was individually assessed at 40, 80, 200, 500, and 1000\,ms.
\end{itemize}

To examine the impact of varying diffusivity and permeability on particle density, additional simulations were conducted using a phantom composed of two substrates, each occupying 50\% of the volume. In these simulations, 5 \(\times\) 10\textsuperscript{4} spins were randomly distributed throughout the phantom and allowed to diffuse for 1 second, with the diffusion trajectory being recorded over the entire simulation period. Three scenarios were examined: 1) both substrates had identical diffusivities, and the membrane between them was bidirectionally permeable, 2) the diffusivity of substrate 1 was twice that of substrate 2, while the membrane remained bidirectionally permeable, and 3) the diffusivities were identical, but particles could only diffuse from substrate 1 to substrate 2 with a 10\% probability (unidirectional permeability), while diffusion in the reverse direction was prohibited.

\subsection*{Performance Benchmarking}

The total number of operations is proportional to \(U = NT\), where \(N\) is the number of walkers and \(T\) is the number of time points (\cite{hall2009convergence}). Given the architecture of SpinWalk, which utilizes a voxel mesh model to represent the numerical phantom, the FoV and grid size are additional factors that can influence computation time. To evaluate the impact of these factors, three phantoms with the following FoV and grid size combinations were generated: [1800\,µm and 600], [600\,µm and 600], and [600\,µm and 1200]. The phantoms were filled with cylinders having a radius of 8\,µm and a total volume fraction of 4\%. Off-resonance was calculated according to equation \(\eqref{eq:cylinder_b0}\) and incorporated into the simulation together with relaxation effects. For each phantom, 10\textsuperscript{5}, 4 \(\times\) 10\textsuperscript{5}, and 10\textsuperscript{6} spins were randomly distributed, and 10\textsuperscript{4} random walks were performed in separate runs. The simulations were executed separately on both GPU and CPU, and the elapsed time was recorded.

With regards to performance benchmarking, a comparison between SpinWalk, simDrift (\cite{blum2023simulated}), and Disimpy (\cite{kerkela2020disimpy}) was conducted for a free diffusion experiment where 10\textsuperscript{6} spins performed 10\textsuperscript{4} random walks. Off-resonance and relaxation effects were excluded in SpinWalk, as these inputs are not considered in simDrift and Disimpy. All three simulators ran on a GPU, and simulations were performed for 2 and 100 different b-values, with the elapsed time recorded. To minimize unnecessary overhead, Disimpy simulations were run in quiet mode, and the simDrift source code was modified to suppress progress display. Disimpy and simDrift employed bipolar diffusion-sensitizing gradients (GRE-based sequence) in the absence of an RF pulse, with the magnetization already flipped. In contrast, SpinWalk used monopolar diffusion-sensitizing gradients (SE-based sequence), incorporating excitation and refocusing pulses, each lasting one time step.
 